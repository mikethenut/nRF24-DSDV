%%%%%%%%%%%%%%%%%%%%%%%%%%%%%%%%%%%%%%%%%
% Beamer Presentation
% LaTeX Template
% Version 1.0 (10/11/12)
%
% This template has been downloaded from:
% http://www.LaTeXTemplates.com
%
% License:
% CC BY-NC-SA 3.0 (http://creativecommons.org/licenses/by-nc-sa/3.0/)
%
%%%%%%%%%%%%%%%%%%%%%%%%%%%%%%%%%%%%%%%%%

%----------------------------------------------------------------------------------------
%	PACKAGES AND THEMES
%----------------------------------------------------------------------------------------

\documentclass{beamer}
\setbeamertemplate{caption}[numbered]

\mode<presentation> {

% The Beamer class comes with a number of default slide themes
% which change the colors and layouts of slides. Below this is a list
% of all the themes, uncomment each in turn to see what they look like.

%\usetheme{default}
%\usetheme{AnnArbor}
%\usetheme{Antibes}
%\usetheme{Bergen}
%\usetheme{Berkeley}
%\usetheme{Berlin}
%\usetheme{Boadilla}
%\usetheme{CambridgeUS}
%\usetheme{Copenhagen}
%\usetheme{Darmstadt}
%\usetheme{Dresden}
%\usetheme{Frankfurt}
%\usetheme{Goettingen}
%\usetheme{Hannover}
%\usetheme{Ilmenau}
%\usetheme{JuanLesPins}
%\usetheme{Luebeck}
\usetheme{Madrid}
%\usetheme{Malmoe}
%\usetheme{Marburg}
%\usetheme{Montpellier}
%\usetheme{PaloAlto}
%\usetheme{Pittsburgh}
%\usetheme{Rochester}
%\usetheme{Singapore}
%\usetheme{Szeged}
%\usetheme{Warsaw}

% As well as themes, the Beamer class has a number of color themes
% for any slide theme. Uncomment each of these in turn to see how it
% changes the colors of your current slide theme.

%\usecolortheme{albatross}
%\usecolortheme{beaver}
%\usecolortheme{beetle}
%\usecolortheme{crane}
%\usecolortheme{dolphin}
%\usecolortheme{dove}
%\usecolortheme{fly}
%\usecolortheme{lily}
%\usecolortheme{orchid}
%\usecolortheme{rose}
%\usecolortheme{seagull}
%\usecolortheme{seahorse}
%\usecolortheme{whale}
%\usecolortheme{wolverine}

%\setbeamertemplate{footline} % To remove the footer line in all slides uncomment this line
%\setbeamertemplate{footline}[page number] % To replace the footer line in all slides with a simple slide count uncomment this line

%\setbeamertemplate{navigation symbols}{} % To remove the navigation symbols from the bottom of all slides uncomment this line
}

\usepackage{graphicx} % Allows including images
\usepackage{booktabs} % Allows the use of \toprule, \midrule and \bottomrule in tables
\usepackage[utf8]{inputenc}
\usepackage{lmodern}

%----------------------------------------------------------------------------------------
%	TITLE PAGE
%----------------------------------------------------------------------------------------

\title[DSDV \& nRF24L01]{Usmerjanje prometa nRF24L01 s protokolom DSDV} % The short title appears at the bottom of every slide, the full title is only on the title page

\author[Mihael Rajh]{Mihael Rajh} % Your name
\institute[BSO projekti 21/22] % Your institution as it will appear on the bottom of every slide, may be shorthand to save space
{
Fakulteta za računalništvo in informatiko, Univerza v Ljubljani \\ % Your institution for the title page
}
\date{20. 5. 2022} % Date, can be changed to a custom date

\begin{document}

\begin{frame}
\titlepage % Print the title page as the first slide
\end{frame}

%\begin{frame}
%\frametitle{Pregled} % Table of contents slide, comment this block out to remove it
%\tableofcontents % Throughout your presentation, if you choose to use \section{} and \subsection{} commands, these will automatically be printed on this slide as an overview of your presentation
%\end{frame}

%----------------------------------------------------------------------------------------
%	PRESENTATION SLIDES
%----------------------------------------------------------------------------------------


\begin{frame}
\frametitle{Uvod}
\begin{itemize}
\item MANET kot možna implementacija WSNET
\item Primerjava MANET in WSNET
\item Tipične težave v MANET omrežjih
\end{itemize}
\end{frame}

\begin{frame}
\frametitle{Usmerjevalni algoritmi}
\begin{itemize}
\item Spremljanje stanja povezav ali vektorjev razdalj
\item Na osnovi tabele ali zahteve
\item Protokol DSDV kot nadgradnja DBF algoritma
\item Sekvenčne številke rešujejo nekatere težave omrežij
\end{itemize}
\end{frame}

\begin{frame}
\frametitle{Modul nRF24L01}
\begin{itemize}
\item Namenjen komunikaciji na kratke razdalje
\item Uporablja 126 kanalov na področju 2,4GHz
\item Komunicira z do 6 napravami hkrati
\item Paketi vsebujejo do 32B podatkov in 3-5B dolge naslove
\end{itemize}
\end{frame}

\begin{frame}
\frametitle{Rešitev: Format podatkov}
\begin{itemize}
\item Dve 32B polji za komunikacijo
\item Posodobitvena tabela z naslovom cilja, naslovom skoka, številom skokov in sekvečno številko
\item Usmerjevalna tabela z oznako spremembe in časom zadnjega prejema
\item Paket 32B = 4x (3B naslov + 4B sekvenca + 1B skoki)
\end{itemize}
\end{frame}

\begin{frame}
\frametitle{Rešitev: Implementirana opravila}
\begin{itemize}
\item DSDV\_init pripravi napravo in modul na delovanje.
\item nRF24\_listen preverja za prispela sporočila.
\item parse\_packet prispela sporočila pravilno prebere.
\item check\_table preverja tabelo za neveljavne vnose.
\item update\_table posodobi usmerjevalno tabelo.
\item format\_packet pripravi spremenjene vrstice za pošiljanje.
\item full\_table\_dump pošlje celotno usmerjevalno tabelo.
\item nRF24\_transmit odda en paket.
\end{itemize}
\end{frame}

\begin{frame}
\frametitle{Rešitev: Uporabniške nastavitve}
\begin{itemize}
\item channel
\item network\_address
\item TABLE\_SIZE\_INIT
\item BRCST\_INTERVAL
\item DUMP\_INTERVAL
\item CHECK\_INTERVAL
\item TIMEOUT
\item ENTRY\_DELETE
\end{itemize}
\end{frame}

\begin{frame}
\frametitle{Zaključek}
\begin{itemize}
\item Problematika MANET kot rešitev WSNET.
\item Predstavljen algoritem DSDV in implementacija za modul nRF24L01.
\item Manjka določanje unikatnega naslova in funkcionalnost protokola.
\item Lahko drugače formatiramo pakete z upoštevanjem overflow sekvenc.
\item Možno zavarovanje protokola in optimizacija delovanja.
\end{itemize}
\end{frame}


\begin{frame}
\Huge{\centerline{END.}}
\end{frame}

%----------------------------------------------------------------------------------------

\end{document} 