%%%%%%%%%%%%%%%%%%%% author.tex %%%%%%%%%%%%%%%%%%%%%%%%%%%%%%%%%%%
%
% sample root file for your "contribution" to a contributed volume
%
% Use this file as a template for your own input.
%
%%%%%%%%%%%%%%%% Springer %%%%%%%%%%%%%%%%%%%%%%%%%%%%%%%%%%%%%%%%%


%% RECOMMENDED %%%%%%%%%%%%%%%%%%%%%%%%%%%%%%%%%%%%%%%%%%%%%%%%%%%
%\documentclass[graybox]{svmult}
%
%% choose options for [] as required from the list
%% in the Reference Guide
%
%\usepackage{mathptmx}       % selects Times Roman as basic font
%\usepackage{helvet}         % selects Helvetica as sans-serif font
%\usepackage{courier}        % selects Courier as typewriter font
%\usepackage{type1cm}        % activate if the above 3 fonts are
                             % not available on your system
%
%\usepackage{makeidx}         % allows index generation
%\usepackage{graphicx}        % standard LaTeX graphics tool
%                             % when including figure files
%\usepackage{multicol}        % used for the two-column index
%\usepackage[bottom]{footmisc}% places footnotes at page bottom
%
%% see the list of further useful packages
%% in the Reference Guide
%
%\makeindex             % used for the subject index
%                       % please use the style svind.ist with
%                       % your makeindex program
%
%%%%%%%%%%%%%%%%%%%%%%%%%%%%%%%%%%%%%%%%%%%%%%%%%%%%%%%%%%%%%%%%%%%%%%%%%%%%%%%%%%%%%%%%%%
%
%\begin{document}
\graphicspath{{Skupina01/img/}}


\title{Usmerjanje prometa z modulom nRF24L01}
% Use \titlerunning{Short Title} for an abbreviated version of
% your contribution title if the original one is too long
\author{Mihael Rajh}
% Use \authorrunning{Short Title} for an abbreviated version of
% your contribution title if the original one is too long

%
% Use the package "url.sty" to avoid
% problems with special characters
% used in your e-mail or web address
%
\maketitle

\abstract*{Abstrakt.}

\abstract{Abstrakt.}

\section{Uvod in motivacija}
Opiši ad-hoc mobilna omrežja in njihove uporabe. Opiši tipične probleme, s katerimi se srečujemo.

 Opiši DVR in link-state pristope. Opiši table-driven in on-demand pristope. Opiši razlog za implementacijo DSDV. 

Opiši idejo za Bellman-Ford, DBF, in DSDV. Opiši DSDV podrobneje.

\section{Uporabljene knjižnice in moduli}
Opiši RF24 modul. Opiši specifične probleme, s katerimi se pri njem srečujemo. Opiši knjižnice, uporabljene pri delu.

\section{Rešitev}
Opiši shemo rešitve: določanje ID, določanje sosedov, izdelava tabel, in pošiljanje/branje sporočil.

Prikaži določanje ID z odsekom kode.

Prikaži določanje sosedov z odsekom kode.

Prikaži izdelavo tabel in odsek kode.

Prikaži branje/pisanje sporočil in odsek kode.

\section{Eksperimenti in rezultati}
Opiši 'dummy' funkcijo za preverjanje pravilnosti. Prikaži rezultate.

\section{Zaključek}
Ponovno opiši temo in algoritem. Opiši zasnovo rešitve in specifike implementacije.

Opiši ideje za praktično uporabo in pristope k testiranju.

Opiši ideje za izboljšavo in primerjavo z drugimi algoritmi.


